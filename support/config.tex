%**************************************************************
% Frontespizio
%**************************************************************

\newcommand{\myName}{Francesco Parolini}

\newcommand{\myTitle}{Simulation-based Inclusion Checking Algorithms for $\omega$-Languages}

\newcommand{\myDegree}{Master Thesis}

\newcommand{\myUni}{Università degli Studi di Padova}

\newcommand{\myFaculty}{Master of Science in Computer Science}

\newcommand{\myDepartment}{Department of Mathematics "Tullio Levi-Civita"}

\newcommand{\profTitle}{Prof.}

\newcommand{\myProf}{Francesco Ranzato}

\newcommand{\myLocation}{Padova}

\newcommand{\myAA}{2019-2020}

\newcommand{\myTime}{July 2020}

%**************************************************************
% Impostazioni di impaginazione
% see: http://wwwcdf.pd.infn.it/AppuntiLinux/a2547.htm
%**************************************************************

\setlength{\parindent}{14pt}   % larghezza rientro della prima riga
\setlength{\parskip}{0pt}   % distanza tra i paragrafi

%**************************************************************
% Impostazioni di biblatex
%**************************************************************
\bibliography{support/bibliography}

\defbibheading{bibliography}{
    \cleardoublepage
    \phantomsection
    \addcontentsline{toc}{chapter}{\bibname}
    \chapter*{\bibname\markboth{\bibname}{\bibname}}
}

\setlength\bibitemsep{1.5\itemsep} % spazio tra entry

\DeclareBibliographyCategory{opere}
\DeclareBibliographyCategory{web}



%**************************************************************
% Impostazioni di caption
%**************************************************************
\captionsetup{
    tableposition=top,
    figureposition=bottom,
    font=small,
    format=hang,
    labelfont=bf
}

%**************************************************************
% Impostazioni di graphicx
%**************************************************************
\graphicspath{{img/}} % cartella dove sono riposte le immagini

%**************************************************************
% Impostazioni di hyperref
%**************************************************************
\hypersetup{
    %hyperfootnotes=false,
    %pdfpagelabels,
    %draft,	% = elimina tutti i link (utile per stampe in bianco e nero)
    colorlinks=true,
    linktocpage=true,
    pdfstartpage=1,
    pdfstartview=FitV,
    % decommenta la riga seguente per avere link in nero (per esempio per la stampa in bianco e nero)
    %colorlinks=false, linktocpage=false, pdfborder={0 0 0}, pdfstartpage=1, pdfstartview=FitV,
    breaklinks=true,
    pdfpagemode=UseNone,
    pageanchor=true,
    pdfpagemode=UseOutlines,
    plainpages=false,
    bookmarksnumbered,
    bookmarksopen=true,
    bookmarksopenlevel=1,
    hypertexnames=true,
    pdfhighlight=/O,
    %nesting=true,
    %frenchlinks,
    urlcolor=webbrown,
    linkcolor=RoyalBlue,
    citecolor=Black,
    citecolor=webgreen,
    %pagecolor=RoyalBlue,
    %urlcolor=Black, linkcolor=Black, citecolor=Black, %pagecolor=Black,
    pdftitle={\myTitle},
    pdfauthor={\textcopyright\ \myName, \myUni, \myFaculty},
    pdfsubject={},
    pdfkeywords={},
    pdfcreator={pdfLaTeX},
    pdfproducer={LaTeX}
}

%**************************************************************
% Impostazioni di xcolor
%**************************************************************
\definecolor{webgreen}{rgb}{0,.5,0}
\definecolor{webbrown}{rgb}{.6,0,0}


%**************************************************************
% Altro
%**************************************************************

\newcommand{\omissis}{[\dots\negthinspace]} % produce [...]

% eccezioni all'algoritmo di sillabazione
\hyphenation
{
    ma-cro-istru-zio-ne
    gi-ral-din
}

\newcommand{\sectionname}{sezione}
\addto\captionsitalian{\renewcommand{\figurename}{Figura}
                       \renewcommand{\tablename}{Tabella}}
\newcommand{\glsfirstoccur}{\ap{{[g]}}}
\newcommand{\intro}[1]{\emph{\textsf{#1}}}


% Reference: https://www.overleaf.com/learn/latex/theorems_and_proofs
\newtheorem{theorem}{Theorem}[section]
\theoremstyle{definition}
\newtheorem{lemma}[theorem]{Lemma}
\newtheorem{corollary}[theorem]{Corollary}
\theoremstyle{definition}
\newtheorem{definition}[theorem]{Definition}
\newtheorem{fact}[theorem]{Fact}
\newtheorem{proposition}[theorem]{Proposition}
\theoremstyle{definition}
\newtheorem{remark}[theorem]{Remark}
\theoremstyle{definition}
\newtheorem{example}[theorem]{Example}

% Commands:
% simulation-based commands: [n][w][p][r|l|s]name[f]
% [n] -> if the relation does not hold
% [w] -> if is a wqo on Words
% [p] -> if parametrized by some constant
% [r|l|s] -> r if "goes on the right", l if "goes on the left", s if is "symmetric"
% [f] -> if has some kind of condition on final states
% Example: wsdirf = preorder on Words, Symmetric, which uses the DIRect simulation and has a condition on Final states

% Direct simulation on states
\newcommand{\dir}[0]{\preceq^{di}}
\newcommand{\ndir}[0]{\npreceq^{di}}
% Reverse simulation on states
\newcommand{\revdir}[0]{\preceq^r}
\newcommand{\nrevdir}[0]{\npreceq^r}
% Backward simulation on states
\newcommand{\bwdir}[0]{\preceq^b}
\newcommand{\nbwdir}[0]{\npreceq^b}
% Delayed simulation on states
\newcommand{\del}[0]{\preceq^{de}}
\newcommand{\ndel}[0]{\npreceq^{de}}
% Fair simulation on states
\newcommand{\fair}[0]{\preceq^{f}}
\newcommand{\nfair}[0]{\npreceq^{f}}

% K-lookahead simulations (transitive or not)
\newcommand{\kldir}[0]{\sqsubseteq^{k-di}}
\newcommand{\kldel}[0]{\sqsubseteq^{k-de}}
\newcommand{\klfair}[0]{\sqsubseteq^{k-f}}
% Transitive version
\newcommand{\tkldir}[0]{\preceq^{k-di}}
\newcommand{\tkldel}[0]{\preceq^{k-de}}
\newcommand{\tklfair}[0]{\preceq^{k-f}}
% Parametrized transitive version
\newcommand{\ptkl}[2]{\preceq^{#1-#2}}

% Trace inclusion on states
\newcommand{\tdir}[0]  {\preceq^{t-di}}
\newcommand{\ntdir}[0] {\npreceq^{t-di}}
\newcommand{\tdel}[0]  {\preceq^{t-de}}
\newcommand{\ntdel}[0] {\npreceq^{t-de}}
\newcommand{\tfair}[0] {\preceq^{t-f}}
\newcommand{\ntfair}[0]{\npreceq^{t-f}}
\newcommand{\pt}[1]{\preceq^{t-#1}}

% Direct right simulation on words
\newcommand{\wrdir}[1]{\sqsubseteq_{\mathcal{#1}}^r}
% Direct left simulation on words
\newcommand{\wldir}[1]{\sqsubseteq_{\mathcal{#1}}^l}
% Direct symmetric simulation on words
\newcommand{\wsdir}[1]{\sqsubseteq^1_{\mathcal{#1}}}
% Direct symmetric simulation on words with condition on final states
\newcommand{\wsdirf}[1]{\sqsubseteq_{\mathcal{#1}}^{2}}
% Dekayed right simulation on words
\newcommand{\wrdel}[1]{\sqsubseteq_{\mathcal{#1}}^{de,r}}
% Fair right simulation on words
\newcommand{\wrfair}[1]{\sqsubseteq_{\mathcal{#1}}^{fair,r}}
% Fair symmetric simulation on words
\newcommand{\wsfair}[1]{\sqsubseteq_{\mathcal{#1}}^{fair}}
% parametrized k-lookahead symmetric simulation on words
\newcommand{\pwskl}[3]{\sqsubseteq_{\mathcal{#1}}^{#2-#3}}
% parametrized trace inclusion symmetric on words
\newcommand{\pwst}[2]{\sqsubseteq_{\mathcal{#1}}^{t-#2}}


% right monotonic state based wqo
\newcommand{\wrs}[1]{\leq_{\mathcal{#1}}^{r}}
% symmetric state based wqo
\newcommand{\wss}[1]{\leq^1_{\mathcal{#1}}}
% symmetric state based wqo, with final states
\newcommand{\wssf}[1]{\leq^2_{\mathcal{#1}}}

\newcommand{\automa}[1]{\mathcal{#1}=\langle Q, \Sigma, \delta, I, F\rangle}
\newcommand{\lang}[1]{\mathcal{L}(\mathcal{#1})}
\newcommand{\goes}[1]{\overset{#1}{\rightsquigarrow}}
\newcommand{\goesUnder}[2]{\underset{#2}{\overset{#1}{\rightsquigarrow}}}
\newcommand{\goesUnderf}[2]{\underset{#2}{\overset{#1}{\rightarrowtail}}}
\newcommand{\goesr}[1]{\overset{#1}{\rightsquigarrow_R}}
\newcommand{\goesf}[1]{\overset{#1}{\rightarrowtail}}
\newcommand{\goesrf}[1]{\overset{#1}{\rightarrowtail_R}}
\newcommand{\trans}[1]{\overset{#1}{\rightarrow}}
\newcommand{\ntrans}[1]{\overset{#1}{\nrightarrow}}
\newcommand{\transr}[1]{\overset{#1}{\rightarrow}_R}
\newcommand{\oldSimBody}{q_1 \preceq q_3 \;\wedge\; q_1 \preceq^{R} q_3 \;\wedge\; q_2 \preceq q_4 \;\wedge\; q_2 \preceq^{R} q_4 }
\newcommand{\trace}[4]{\pi_#1 = #2_0 \trans{#3_0} #2_1 \trans{#3_1} \cdots \trans{#3_{#4-1}} #2_#4}
\newcommand{\traceNoName}[3]{\pi = #1_0 \trans{#2_0} #1_1 \trans{#2_1} \cdots \trans{#2_{#3-1}} #1_#3}
\newcommand{\transs}[3]{\underbrace{\trans{#1} \cdots \trans{#2}}_{#3}}
\newcommand{\varstate}[5]{q^{#2}_{#1^{#3}_{#4}#5}}
\newcommand{\traceState}[2]{q^{#1}_{#2}}
\newcommand{\istate}[2]{\varstate{i}{#1}{#1}{#2}{}}
\newcommand{\istateMod}[3]{\varstate{i}{#1}{#1}{#2}{#3}}

\newcommand{\powerset}{\raisebox{.10\baselineskip}{\ensuremath{\wp}}}
\newcommand{\fpowerset}{\raisebox{.10\baselineskip}{\ensuremath{\wp}}_f}
\newcommand{\lfp}{\textrm{lfp}\;}
\newcommand{\plfp}[1]{\textrm{lfp}#1\;}
\newcommand{\gfp}{\textrm{gfp}\;}
\newcommand{\pgfp}[1]{\textrm{gfp}#1\;}
\newcommand{\eqdef}{\overset{\triangle}{=}}
\newcommand{\iffdef}{\overset{\triangle}{\Longleftrightarrow}}
\newcommand{\angles}[1]{\langle #1 \rangle}
\newcommand{\Di}[1]{D_{1,#1}}
\newcommand{\Din}[2]{D_{1,#1}^{#2}(\emptyset)}
% D initial to the n, with element
\newcommand{\Dine}[3]{(D_{1,#1}^{#2}(\emptyset))_{#3}}
\newcommand{\Dp}[1]{D_{2,#1}}
\newcommand{\Dpn}[2]{D_{2,#1}^{#2}(\emptyset)}
\newcommand{\Dpne}[3]{(D_{2,#1}^{#2}(\emptyset))_{#3}}
\newcommand{\posti}[2]{post_{#1}^{\mathcal{#2}}(I)}
\newcommand{\alg}[3]{\mathscr{A}_{#1, #2}(#3)}
\newcommand{\ctx}[2]{ctx_{\mathcal{#1}}(#2)}
\newcommand{\ctxf}[2]{ctx^F_{\mathcal{#1}}(#2)}
\newcommand{\gfunc}{\vec{b} \cup Fn_{\mathcal{G}}(\vec{X})}
\newcommand{\gfunclambda}{\lambda \vec{X}. (\vec{b} \cup Fn_{\mathcal{G}}(\vec{X}))}
% \newcommand{\gc}[4]{\angles{#1,#2} \overset{\alpha}{\underset{\gamma}{\leftrightarrows}} \angles{#3,#4}}
\newcommand{\gc}[4]{\angles{#1,#2} \galois{\alpha}{\gamma} \angles{#3,#4}}
\newcommand{\pgc}[6]{\angles{#1,#2} \galois{#5}{#6} \angles{#3,#4}}
\newcommand{\minor}[1]{\lfloor #1 \rfloor}
\newcommand{\powersetSigmas}{\powerset (\Sigma^*)}
\newcommand{\acSigmas}{AC_{\angles{\Sigma^*, \leq}}}
\newcommand{\omegaIncName}{\texttt{BAInc}}
\newcommand{\refOmegaInc}{\hyperref[alg:omega-lang-inc]{\omegaIncName}}
\newcommand{\prefixesName}{\texttt{BAPrefixes}}
\newcommand{\refPrefixes}{\hyperref[alg:prefixes]{\prefixesName}}
\newcommand{\periodsName}{\texttt{BAPeriods}}
\newcommand{\refPeriods}{\hyperref[alg:periods]{\periodsName}}
\newcommand{\grammarName}{\texttt{CFGInc}}
\newcommand{\refGrammar}{\hyperref[alg:grammar]{\grammarName}}
\renewcommand{\thealgocf}{}
\newcommand{\Buchi}{B{\"u}chi}
\newcommand{\lambdaRegular}{\lambda X. b \cup Fn(X)}
